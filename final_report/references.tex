% !TEX encoding = IsoLatin
\begin{thebibliography}{9}
\addcontentsline{toc}{section}{Riferimenti bibliografici}

\bibitem{defautism}
Gene Blatt,
``autism'',
Encyclopedia Britannica,
9 Sep. 2021,
\url{https://www.britannica.com/science/autism},
Accessed 27 October 2021.

\bibitem{abide}
Child Mind Institute,
``Autism Brain Imaging Data Exchange'',
ABIDE,
Child Mind Institute,
\url{http://fcon_1000.projects.nitrc.org/indi/abide/},
Accessed 27 October 2021.

\bibitem{sexdiff}
Kaat Alaerts, Stephan P. Swinnen, and Nicole Wenderoth.,
``Sex Differences in Autism: A Resting-State Fmri Investigation of Functional Brain Connectivity in Males and Females.'',
Social cognitive and affective neuroscience.,
U.S. National Library of Medicine.,
Accessed October 27, 2021.,
\url{https://pubmed.ncbi.nlm.nih.gov/26989195/}.

\bibitem{preprocabide}
Cameron Craddock, Pierre Bellec.,
``ABIDE Preprocessed.'',
Preprocessed Connectomes Project.,
Accessed October 27, 2021.,
\url{http://preprocessed-connectomes-project.org/abide/}.

\bibitem{neuroimg}
Cameron Craddock, Yassine Benhajali, Carlton Chu, Francois Chouinard, Alan Evans, András Jakab, Budhachandra Singh Khundrakpam, John David Lewis, Qingyang Li, Michael Milham, Chaogan Yan, Pierre Bellec,
``The Neuro Bureau Preprocessing Initiative: open sharing of preprocessed neuroimaging data and derivatives.'',
In Neuroinformatics 2013,
Stockholm, Sweden,
\doi{10.3389/conf.fninf.2013.09.00041}.

\bibitem{abide2}
Child Mind Institute.,
``Autism Brain Imaging Data Exchange II'',
ABIDE. Child Mind Institute., \url{http://fcon_1000.projects.nitrc.org/indi/abide/abide_II.html}.,
Accessed 27 October 2021.

\bibitem{guidelinesml}
Pegah Kassraian-Fard, Caroline Matthis, Joshua H. Balsters, Marloes H. Maathuis, Nicole Wenderoth,
``Promises, Pitfalls, and Basic Guidelines for Applying Machine Learning Classifiers to Psychiatric Imaging Data, with Autism as an Example'',
Frontiers in Psychiatry,
vol. 7,
2016,
ISSN-1664-0640,
\doi{10.3389/fpsyt.2016.00177}

\bibitem{neuroml}
Abraham, Alexandre and Pedregosa, Fabian and Eickenberg, Michael and Gervais, Philippe and Mueller, Andreas and Kossaifi, Jean and Gramfort, Alexandre and Thirion, Bertrand and Varoquaux, Gael,
``Machine learning for neuroimaging with scikit-learn'',
Frontiers in Neuroinformatics,
vol. 8,
2014,	
\url{https://www.frontiersin.org/article/10.3389/fninf.2014.00014},
ISSN 1662-5196,
Statistical machine learning methods are increasingly used for neuroimaging data analysis. Their main virtue is their ability to model high-dimensional datasets, e.g., multivariate analysis of activation images or resting-state time series. Supervised learning is typically used in decoding or encoding settings to relate brain images to behavioral or clinical observations, while unsupervised learning can uncover hidden structures in sets of images (e.g., resting state functional MRI) or find sub-populations in large cohorts. By considering different functional neuroimaging applications, we illustrate how scikit-learn, a Python machine learning library, can be used to perform some key analysis steps. Scikit-learn contains a very large set of statistical learning algorithms, both supervised and unsupervised, and its application to neuroimaging data provides a versatile tool to study the brain,
\doi{10.3389/fninf.2014.00014}

\bibitem{fmriprep}
``fMRIPrep: A Robust Preprocessing Pipeline for fMRI Data'',
\url{https://fmriprep.org/en/stable/}

\bibitem{andysbrainbook}
\url{https://andysbrainbook.readthedocs.io/}

\bibitem{cpac}
``Configurable Pipeline for the Analysis of Connectomes (C-PAC)'',
\url{https://fcp-indi.github.io/}

\bibitem{repeated_kfold}
    ``Repeated k-fold Cross Validation with python'',
    \url{https://machinelearningmastery.com/repeated-k-fold-cross-validation-with-python/}


\end{thebibliography}